%------------------------------------
% Dario Taraborelli
% Typesetting your academic CV in LaTeX
%
% URL: http://nitens.org/taraborelli/cvtex
% DISCLAIMER: This template is provided for free and without any guarantee 
% that it will correctly compile on your system if you have a non-standard  
% configuration.
% Some rights reserved: http://creativecommons.org/licenses/by-sa/3.0/
%------------------------------------


%!TEX TS-program = xelatex
%!TEX encoding = UTF-8 Unicode

%\documentclass[11pt, a4paper]{article}
\documentclass[11pt, letterpaper]{article}
%\documentclass[10pt, letterpaper]{article}
\usepackage{fontspec} 

% DOCUMENT LAYOUT
\usepackage{geometry} 
%\geometry{a4paper, textwidth=5.5in, textheight=8.5in, marginparsep=7pt, marginparwidth=.6in}
\geometry{letterpaper, textwidth=5.5in, textheight=9.0in, marginparsep=7pt, marginparwidth=.6in}
\setlength\parindent{0in}

% FONTS
\usepackage{xunicode}
\usepackage{xltxtra}
\defaultfontfeatures{Mapping=tex-text} % converts LaTeX specials (``quotes'' --- dashes etc.) to unicode
\setromanfont [Ligatures={Common},Numbers={OldStyle}]{Adobe Caslon Pro}
\setmonofont[Scale=0.8]{Monaco} 
\setsansfont[Scale=0.9]{Optima Regular} 
% ---- CUSTOM AMPERSAND
\newcommand{\amper}{{\fontspec[Scale=.95]{Adobe Caslon Pro}\selectfont\itshape\&}}
% ---- MARGIN YEARS
\usepackage{marginnote}
\newcommand{\years}[1]{\marginnote{\scriptsize #1}}
\renewcommand*{\raggedleftmarginnote}{}
\setlength{\marginparsep}{7pt}
\reversemarginpar

\usepackage[misc]{ifsym} %adds email symbol ('\Letter')
\usepackage{marvosym}  %adds cell phone symbol ('\Mobilefone')
\usepackage{changepage} %allows left/right indentation of body text

% HEADINGS
\usepackage{sectsty} 
\usepackage[normalem]{ulem} 
\sectionfont{\rmfamily\mdseries\Large} 
\subsectionfont{\rmfamily\mdseries\scshape\normalsize} 
\subsubsectionfont{\rmfamily\bfseries\upshape\normalsize} 

% PDF SETUP
% ---- FILL IN HERE THE DOC TITLE AND AUTHOR
\usepackage[xetex, bookmarks, colorlinks, breaklinks, pdftitle={john_ortiz_cv},pdfauthor={John P. Ortiz}]{hyperref}  
%\hypersetup{linkcolor=blue,citecolor=blue,filecolor=black,urlcolor=blue} 
%\hypersetup{linkcolor=black,citecolor=blue,filecolor=black,urlcolor=black} 
\hypersetup{linkcolor=black,citecolor=blue,filecolor=black,urlcolor=blue} 

% define a double line separator
\def\doubleline{

	\vspace{-1.4em}
	\hspace{\fill}\linethickness{0.7pt}\line(1,0){5.5in}\hspace{\fill}
	
	\vspace{-1.0em}
	\hspace{\fill}\linethickness{0.7pt}\line(1,0){5.5in}\hspace{\fill}
	
}

% CENSORING PRIVATE INFO
% use \censor{} to black out pieces of text
% use \StopCensoring to turn off all instances of censoring
\usepackage{censor}

% MULTIPLE COLUMNS
\usepackage{multicol}
%-----------------------------------------------------------------------------------

% DOCUMENT
\begin{document}
{\LARGE John P. Ortiz} %\\[1cm]
%{\Huge John P. Ortiz} %\\[1cm]
%\vfill

%% centered info with double line underneath
%\begin{center}
%	{\LARGE John P. Ortiz} \\[0.5cm]
%	799 6th St. \#14, Los Alamos, NM 87544 •\ (541) 207-5846 •\ john.p.ortiz.14@gmail.com
%\end{center}
%\doubleline

% \StopCensoring     %<------------------ UNCOMMENT TO TURN OFF CENSORING
\begin{center}
	\begin{tabular}{l l}
		Ph.D. Student    &
		\hspace{.5in}\href{www.lanl.gov/EES-16}{www.lanl.gov/EES-16} \\ Johns
		Hopkins University   &
		\hspace{.5in}\href{http://johnportiz.weebly.com/}{www.johnportiz.weebly.com}
		\\
		Whiting School of Engineering       & \hspace{.5in}\Letter {     }
		\href{mailto:john.p.ortiz.14@gmail.com}{john.p.ortiz.14@gmail.com}
		\\ 	
		Department of Environmental Health \& Engineering  &
		\hspace{.5in}\Mobilefone { } \censor{+1 (541) 207-5846}  \\
%		Ph.D. Graduate Research Assistant    &
%		\hspace{.5in}\href{www.lanl.gov/EES-16}{www.lanl.gov/EES-16} \\
%		Los Alamos National Laboratory   &
%		\hspace{.5in}\href{http://johnportiz.weebly.com/}{www.johnportiz.weebly.com}
%		\\
%		Computational Earth Science Group (EES-16)       & \hspace{.5in}\Letter
%		{     }
%		\href{mailto:john.p.ortiz.14@gmail.com}{john.p.ortiz.14@gmail.com} \\ 
%		\censor{3215 N. Charles St. Apt. #107, Baltimore, MD 21218} &
%		\hspace{.5in}\Mobilefone { } \censor{+1 (541) 207-5846}  \\
		%\censor{5015 Carriage House,   Los Alamos, NM 87544} &
		%\hspace{.5in}\Mobilefone { } \censor{+1 (541) 207-5846}  \\		
	\end{tabular}
\end{center}

%%\hrule
%\section*{Current position}
%\emph{Emeritus Professor}, Institute for Advanced Study, Princeton

%-----------------------AREAS OF SPECIALIZATION-----------------------
%%\hrule
\section*{Areas of Specialization}
% 	numerical modeling, hydrogeology, gas flow in fractured media, petroleum
% 	geofluids
	numerical modeling •\ hydrogeology •\ gas flow in fractured media •\
	petroleum geofluids
	

%--------------------------EDUCATION-------------------------
\section*{Education}
\noindent

\years{Current}\textsc{Johns Hopkins University}
\begin{adjustwidth}{5pt}{0pt}
	\textsc{Ph.D.} Student in Environmental Health \& Engineering \
	%\textsc{Thesis:} The role of fault-zone architectural elements and basal
	%altered zones on downward pore pressure propagation and induced seismicity
	%in the crystalline basement
\end{adjustwidth}

\years{2017}\textsc{New Mexico Institute of Mining and Technology (NMT)}
\begin{adjustwidth}{5pt}{0pt}
	\textsc{M.Sc.} in Hydrology\\ %, \textsc{    GPA:} 3.89/4.00\\
	\textsc{Thesis:} The role of fault-zone architectural elements and basal
	altered zones on downward pore pressure propagation and induced seismicity
	in the crystalline basement
\end{adjustwidth}

\years{2014}\textsc{Dartmouth College}
\begin{adjustwidth}{5pt}{0pt}
	\textsc{B.A.} in Earth Sciences with Honors\\
	\textsc{Honors Thesis:} Quantifying regional sediment flux from
	observations of nearshore morphology in the Columbia River Littoral Cell 
\end{adjustwidth}


%\years{2017}\textsc{M.Sc.} in Hydrology, New Mexico Institute of Mining and Technology (NMT)
%\begin{adjustwidth}{5pt}{0pt}
%
%	\textsc{Thesis:} The Role of Fault-Zone Architectural Elements and Basal Altered Zones on Downward Pore Pressure Propagation and Induced Seismicity in the Crystalline Basement
%%	\begin{tabular}{@{}l l}
%%	\textsc{Advisor:} Dr. Mark Person & \hspace{1cm} \textsc{GPA:} 3.89/4.00 
%	 \textsc{GPA:} 3.89/4.00
%%\end{tabular}\\
%\end{adjustwidth}
%
%
%\years{2014}\textsc{B.A.} in Earth Sciences, Dartmouth College\
%\begin{adjustwidth}{5pt}{0pt}
%	\textsc{Honor's Thesis:} Quantifying regional sediment flux from observations of nearshore morphology in the Columbia River Littoral Cell 
%\end{adjustwidth}

%--------------------------HONORS/AWARDS------------------------------
%\hrule
%\section*{Grants, honors \amper{} awards}
\section*{Honors \amper{} Awards}
\noindent

%\years{2019 - 2020}Nicholas P. Jones Fellowship, Whiting School of Engineering, Johns Hopkins University
%\begin{adjustwidth}{5pt}{0pt}
%\end{adjustwidth}

\years{2018}Top 20 Most Downloaded Recent Papers, Wiley Publishing (\emph{Geofluids})
\begin{adjustwidth}{5pt}{0pt}
	Amongst articles published between July 2016 and June 2018, the article ``Exploring the potential linkages..." (see \hyperref[sec:pubs]{Publications \& Presentations}) was in the top 20 for number of downloads in the 12-months post online publication.
\end{adjustwidth}

\years{2018}“Spot" Performance Award (May), Los Alamos National Laboratory
\begin{adjustwidth}{5pt}{0pt}
	Going above and beyond the call of duty under tight and/or emergency deadlines to finish a project deliverable in the form of a software package that rapidly predicts fractured rock gas transport to the earth’s surface.
\end{adjustwidth}
%-----------------------RESEARCH EXPERIENCE---------------------------
\section*{Research Experience}
\noindent
\years{2020 - Present}\textsc{Doctoral Student, GRA (Graduate Research Assistant)}\\
\textit{Computational Earth Science Group (EES-16), Los Alamos National Laboratory}
\begin{adjustwidth}{5pt}{0pt}
	Carrying out programmatic work for governments sponsors related to detecting
	and verifying underground nuclear explosion (UNE) tests in addition to
	performing research tasks for Ph.D. Some  continuation of research and
	responsibilities from Post-Master's Student appointment (see below). 
\end{adjustwidth}

\years{2017 - 2019}\textsc{Post-Master's Student, GRA}\\
\textit{Computational Earth Science Group (EES-16), Los Alamos National Laboratory}
\begin{adjustwidth}{5pt}{0pt}
	Developed numerical approaches for detecting and verifying underground
	nuclear explosion (UNE) tests. Projects includes simulating
	radionuclide gas transport in fractured geologic media using finite-element
	method (FEM) and control volume finite-element (CVFEM) numerical models,
	simulating high-pressure methane injection into shale samples to inform
	laboratory investigations, determining laboratory- and field-scale
	transport properties of rocks using models and tracer experiments, and
	developing a reduced-order model (ROM) for rapid prediction of gas seepage
	times.\\
	Developed the Amanzi high performance computing (HPC) flow \& transport
	simulator to meet the Nuclear Quality Assurance-1 (NQA-1) regulatory
	standard by improving code verification and benchmark tests. Collaborated
	on a multi-lab program supported by the DOE Office of Environmental
	Management (DOE EM) to provide scientifically defensible and standardized
	assessments of the uncertainties and risks associated with the
	environmental cleanup and closure of waste sites.
\end{adjustwidth}

\years{2016 - 2017}\textsc{Graduate Research Assistant}\\
\textit{Earth and Environmental Sciences Department, NMT}
\begin{adjustwidth}{5pt}{0pt}
	Created transient 3D finite-difference (FDM) models in MODFLOW to analyze
	fluid-fault interactions as pertaining to a suite of basal reservoir
	injection scenarios. I also developed transient 2D cross-sectional FDM
	models in MATLAB to test fluid-fault interactions for crystalline basement
	fault zones exhibiting local, dynamically enhanced permeability caused by
	excess fluid pressures. In my work, I developed new approaches for
	representing fault zones with multiple architectural components and
	identified several key hydrogeologic parameters that control deep
	propagation of the fluid pressure envelope and thus present increased risk
	of induced seismic events.\\
	Collaborated on an NSF-funded (via NM EPSCoR) project deploying subsurface
	field survey equipment (transverse electromagnetics [TEM], magnetotellurics
	[MT]) for interpretation of deep saline geothermal flow regimes in order to
	evaluate potential hydrothermal systems in southern New Mexico.
\end{adjustwidth}

\years{2013}\textsc{Research Intern}\\
\textit{College of Earth, Ocean, and Atmospheric Sciences, Oregon State University}
\begin{adjustwidth}{5pt}{0pt}
	Completed an NSF-funded Research Experience for Undergraduates (REU)
	internship on coastal processes by collecting and interpolating nearshore
	topographic and bathymetric data to extract key spatial and temporal
	metrics using MATLAB. Determined rates of longshore-uniform sandbar
	migration cycles along the Oregon and Washington coasts representing a huge
	component of seasonal coastal sediment flux.
\end{adjustwidth}

\years{2013}\textsc{Research Intern}\\
\textit{United States Army Corps of Engineers Field Research Facility, Duck NC}
\begin{adjustwidth}{5pt}{0pt}
	Studied coastal evolution and erosion by collecting and processing LiDAR
	observations during and after storm surges.\\
	Created a paleo-hurricane record of St. Croix by performing grain-size
	analysis on sediment cores combined with charcoal carbon dating.
\end{adjustwidth}

%-----------------------TEACHING EXPERIENCE---------------------------
\section*{Teaching Experience}
\noindent
\years{2019}\textsc{Grading Assistant}\\
\textit{Environmental Health \& Engineering Department, Johns Hopkins University}
\begin{adjustwidth}{5pt}{0pt}
	Graded weekly homework assignments for 20+ upper-level undergraduate
	engineering students in an Introduction to Fluid Mechanics course.
	Occasionally taught course lectures as needed.
\end{adjustwidth}
\years{2015 - 2016}\textsc{Graduate Teaching Assistant}\\
\textit{Earth and Environmental Sciences Department, NMT}
\begin{adjustwidth}{5pt}{0pt}
	Developed lesson plans and led class lectures and field trips for 20+ students in intermediate and upper level undergraduate Earth Sciences lab sections. Taught lab sections for courses in Geomorphology and Stratigraphy \& Paleontology with work load totaling 20 hrs/week.  
\end{adjustwidth}

%--------------------------SKILLS-------------------------------
\section*{Skills}
\subsection*{Programming/command languages}
\noindent
\textsc{Python} – High proficiency in designing and executing sensitivity analyses of FEHM and PFLOTRAN numerical models and generating nodal grids for finite-element meshes. Have also developed a standalone software application for real-time predictions of gas breakthrough for use on Windows, Linux, and MacOS platforms.\\[5pt]
\textsc{Matlab} – High proficiency in writing scripts for data processing and developing a variety of iterative/numerical models to solve hydrogeologic environmental problems.\\[5pt]
\textsc{R} – Moderate proficiency in building a variety of statistical models to interpret and synthesize climatic, hydrologic, geochemical, and environmental data. Collaborated on statewide New Mexico water budget research project predicting reference evapotranspiration using different statistical models on a wide range remote sensing parameters.\\[5pt]
\LaTeX – Moderate proficiency in typesetting/document preparation of technical and scientific documentation. \\[5pt]
\textsc{Bash/shell}
\subsection*{Software, codes}
\noindent
\textsc{Fehm} – High proficiency in creating multi-phase fluid flow simulations with mass transport. Created test suites comparing vapor tracer transport within a fracture with matrix diffusion to an analytical solution governed by a modified Richard’s equation for unsaturated flow (with immobile liquid phase) in order to test when multi-phase flow significantly affects gas tracer transport. \href{https://github.com/lanl/FEHM}{github.com/lanl/FEHM} \\[5pt]
\textsc{Pflotran} – Intermediate proficiency in creating multi-phase fluid flow simulations with mass transport. Created test suites in similar capacity to work done with FEHM. Also used in conjunction with FEHM to verify the capabilities of discrete fracture networks (DFNs) to simulate flow/transport in fractured media under transient conditions. \href{https://www.pflotran.org}{www.pflotran.org} \\[5pt]
\textsc{LaGriT} – Intermediate proficiency in generating numerical finite-element meshes for a variety of geological applications, including porous flow and transport modeling and discrete fracture networks. Proficient at generating meshes using the 3 interfaces: command line, batch driven via control file, and PyLaGrit Python command/batch interface. Have also worked on capability development by using novel approaches to combine continuum 3D grids with 2D discrete fracture network (DFN) planes into unified meshes for use in transient flow and transport models. \href{https://lagrit.lanl.gov/index.shtml}{lagrit.lanl.gov/index.shtml} \\[5pt]
\textsc{Modflow} – High proficiency in developing 3D transient and steady-state FDM models to solve a variety of hydrogeologic problems including fluid-fault interactions as related to induced seismicity in thesis work, and basin-scale backwards modeling of groundwater solute transport as related to the Kirtland Air Force Base spill in Albuquerque, NM.\\[5pt]
\textsc{Sphinx} – High proficiency in generating and maintaining professional documentation of software projects as HTML and \LaTeX \ output from reStructuredText sources.\\[5pt]
\textsc{Git} – Moderate proficiency in source code management/version control in application development and collaborating with colleagues on coding/software projects. Have experience maintaining Git repositories as a lead developer as well as in a supporting collaborator role.\\[5pt]
\textsc{ParaView} – Moderate proficiency in interactive scientific visualization and exploration by representing data in static figures and videos. \href{https://www.paraview.org/}{www.paraview.org} \\[5pt]
\textsc{Petromod} – Moderate proficiency in creating basin evolution, subsidence rate, and heat flow models of several American petroleum-bearing locales.\\[5pt] 
\textsc{ArcGIS} – Moderate proficiency in using digital elevation models (DEM) to delineate watersheds and create estimates of stream discharge in mountainous terrain.\\[5pt] 
\textsc{Adobe Illustrator} – High proficiency in creating and editing publication-ready figures and illustrative schematic diagrams.\\[5pt] 
\textsc{COMSOL Multiphysics} – Beginner proficiency in modeling contaminant transport in academic coursework.\\[5pt] 
\textsc{Microsoft Excel} – Ranked \#7 in the world at Microsoft Office Worldwide Competition 2008 in Hawaii of 56,000 initial entrants (www.betheworldchamp.com), Ireland National Champion (of 436 entrants).
\subsection*{Other}
\noindent
Advanced proficiency in written and conversational Spanish (nine years of
academic study).


%--------------------------PUBLICATIONS--------------------------
\section*{Publications \amper{} Presentations}
\label{sec:pubs}  %label for linking to sections

%\subsection*{Journal articles \amper{} Reports}
\subsection*{Journal articles \& Reports}
\noindent
\years{2020}Bourret, S. M., Kwicklis, E. M., Harp, D. R., \textbf{Ortiz, J.
P.}, Stauffer, P. H. Beyond Barnwell: Applying lessons learned from the
Barnwell site to other historic underground nuclear tests at Pahute Mesa to
understand radioactive gas-seepage observations. \emph{Journal of Environmental
Radioactivity} (in review).\\
\years{2020}Petrie, E. S., Bradbury, K. K., Cuccio, L., Smith, K., Evans, J.
P., \textbf{Ortiz, J. P.}, Kerner, K., Person, M. A., Mozley, P. S.
Geologic characterization of nonconformities using outcrop and whole-rock core
analogues: hydrologic implications for injection-induced seismicity.
\emph{Solid Earth Discussions}. https://doi.org/10.5194/se-2020-20 (in review).\\
\years{2019}Harp, D. R., \textbf{Ortiz, J. P.}, Stauffer, P. H. Identification
of dominant gas transport frequencies during barometric pumping of fractured
rock. \emph{Scientific Reports (Nature Publishing Group)}, 9(1), 9537.
doi:10.1038/s41598-019-46023-z. \\ 
\years{2019}\textbf{Ortiz, J. P.}, Person,
M. A., Mozley, P. S., Evans, J. P., Bilek, S. L. The role of fault-zone
architectural elements on pore pressure propagation and induced seismicity.
\emph{Groundwater}, 57(3): 465-478. doi:10.1111/gwat.12818\\
\years{2019}Stauffer, P. H., Rahn, T., \textbf{Ortiz, J. P.}, Salazar, L. J.,
Boukhalfa, H., Behar, H. R., Snyder, E. E. Evidence for High Rates of Gas
Transport in the Deep Subsurface. \emph{Geophysical Research Letters}.
doi:10.1029/2019GL082394.\\ 
\years{2018}Harp, D. R., \textbf{Ortiz, J. P.},
Pandey, S., Karra, S., Anderson, D., Bradley, C., Viswanathan, H., \& Stauffer,
P. H. Immobile Pore-Water Storage Enhancement and Retardation of Gas Transport
in Fractured Rock. \emph{Transport in Porous Media}, 1-26.
doi:10.1007/s11242-018-1072-8\\ 
\years{2018}Stauffer, P. H., Rahn, T. A.,
\textbf{Ortiz, J. P.}, Salazar, L. J., Boukhalfa, H., \& Snyder, E. E. Summary
of a Gas Transport Tracer Test in the Deep Cerros Del Rio Basalts, Mesita del
Buey, Los Alamos NM. United States. doi:10.2172/1417180.\\
\years{2017}\textbf{Ortiz, J. P.} The Role of Fault-Zone Architectural Elements
and Basal Altered Zones on Downward Pore Pressure Propagation and Induced
Seismicity in the Crystalline Basement. \emph{New Mexico Institute of Mining
and Technology.}\\ 
\years{2016}Zhang, Y., Edel, S. S., Pepin, J., Person, M.,
Broadhead, R., \textbf{Ortiz, J. P.}, Bilek, S. L., Mozley, P. S., \& Evans, J.
P. (2016). Exploring the potential linkages between oil-field brine
reinjection, crystalline basement permeability, and triggered seismicity for
the Dagger Draw Oil field, southeastern New Mexico, USA, using hydrologic
modeling. \emph{Geofluids}. https://doi.org/10.1111/gfl.12199\\
\years{2014}\textbf{Ortiz, J. P.} Quantifying regional sediment flux from
observations of nearshore morphology in the Columbia River Littoral Cell.
\emph{Dartmouth College Senior Honors Thesis Collection.}\\ 
\years{2014}Cohn, N., Ruggiero, P., \textbf{Ortiz, J. P.}, \& Walstra, D. J.
Investigating the role of complex sandbar morphology on nearshore
hydrodynamics. \emph{Journal of Coastal Research}, 70(sp1), 53-58.
\\
%\subsection*{Conference Talks \amper{} Posters}
\subsection*{Conference Talks \& Posters}
\years{2018}\textbf{Ortiz, J. P.}, Harp, D. R., Stauffer, P. H., Gable, C. W.,
Makedonska, N. Coupled discrete fracture and 3D continuum domain representation
to efficiently capture gas transport from underground cavities, poster
presentation. AGU 2018 Fall Meeting in Washington D.C.\\
\years{2018}Harp, D. R., \textbf{Ortiz, J. P.}, Kwicklis, E. M., Bourret, S.
M., Viswanathan, H., Stauffer, P. H. Identifying dominant barometric
frequencies driving gas transport in fractured rock, poster presentation
(presenting for Dylan Harp). AGU 2018 Fall Meeting in Washington D.C.\\
\years{2018}\textbf{Ortiz, J. P.}, Harp, D. R., Stauffer, P. H. A reduced-order
model to assist real-time predictions of gas transport in unsaturated fractured
media, oral presentation. InterPore 10th Annual Meeting in New Orleans (May
2018).\\
\years{2018}Harp, D. R., \textbf{Ortiz, J. P.}, Stauffer, P. H., Viswanathan,
H., Anderson, D. N., Bradley, C. Analysis of enhanced gas transport in
fractured rock due to barometric pressure variations, oral presentation
(presenting for Dylan Harp). InterPore 10th Annual Meeting in New Orleans (May
2018).\\
\years{2017}\textbf{Ortiz, J. P.}, Ortega, A. D., Harp, D. R., Boukhalfa, H.
Improving estimates of subsurface gas transport in unsaturated fractured media
using field tracer data and numerical methods, oral presentation. AGU 2017 Fall
Meeting in New Orleans.\\
\years{2016}\textbf{Ortiz, J. P.}, Person, M. A., Mozley, P. S., Evans, J. P.
The Hydrologic Connection Between Basal Reservoir Injection, Crystalline
Basement Fault Zones, and Induced Seismicity, oral presentation. September 2016
GSA Annual Meeting in Denver, CO.\\
\years{2013}\textbf{Ortiz, J. P.}, Ruggiero, P., Cohn, N. Interannual Sandbar
Variability within the Columbia River Littoral Cell, poster presentation at the
2013 AGU Fall Meeting in San Francisco.


%\subsection*{Newspaper articles}
%\noindent
%\years{1940}Einstein, Albert, et al. (December 4, 1948), “To the editors", \emph{New York Times}\\
%\years{1949}Einstein, Albert (May 1949), “Why Socialism?", \emph{Monthly Review}.
%

%--------------------------PROFESSIONAL AFFILIATIONS--------------------------
\section*{Professional Affiliations}
\years{2013 - present} \textsc{American Geophysical Union (AGU)}\\
\years{2015 - present} \textsc{American Association of Petroleum Geologists (AAPG)}\\
\years{2016 - present} \textsc{Geological Society of America (GSA)}\\
\years{2017 - present} \textsc{National Ground Water Association (NGWA)}\\
\years{2017 - present} \textsc{New Mexico Geological Society (NMGS)}\\
\years{2018 - present} \textsc{International Society for Porous Media (InterPore)}

%--------------------------SERVICE TO PROFESSION-------------------------
% or "Administrative and Collective Duties"
% - journals reviewed for, funding bodies you act as an expert for, evaluation committees on which you sit
\section*{Service to Profession}
%\subsection*{Journal Reviewer}\noindent
%\textit{Geophysical Research Letters}\\
%\textit{Hydrogeology Journal}

\subsection*{Journal Reviewer}
%\vspace{-0.9cm}
\setlength\multicolsep{0pt} %reduce vertical white space at beginning of cols
\begin{multicols}{3}
	\textit{Geophysical Research Letters}\\
	\textit{Hydrogeology Journal}
\end{multicols}
\subsection*{Advisory Roles}\noindent

\years{2018 - 2019} \textit{LANL Student Programs Advisory Committee (SPAC)}\\
Served as a 1-year appointment as Graduate Student Representative to a committee that advises Los Alamos National Laboratory's Student Programs Office on matters related to the hiring and general well-being of student employees.



%\vspace{1cm}
\vfill{}
%\hrulefill

\begin{center}
{\scriptsize  Last updated: \today\- •\- Typeset in \href{http://nitens.org/taraborelli/cvtex}{
\fontspec{Times New Roman}\XeTeX }}
\end{center}
% ---- FILL IN THE FULL URL TO YOUR CV HERE
%\href{http://nitens.org/taraborelli/cvtex}{http://nitens.org/taraborelli/cvtex}}
%\end{center}

\end{document}
