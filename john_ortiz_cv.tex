%------------------------------------
% Dario Taraborelli
% Typesetting your academic CV in LaTeX
%
% URL: http://nitens.org/taraborelli/cvtex
% DISCLAIMER: This template is provided for free and without any guarantee 
% that it will correctly compile on your system if you have a non-standard  
% configuration.
% Some rights reserved: http://creativecommons.org/licenses/by-sa/3.0/
%------------------------------------


%!TEX TS-program = xelatex
%!TEX encoding = UTF-8 Unicode

%\documentclass[11pt, a4paper]{article}
\documentclass[11pt, letterpaper]{article}
%\documentclass[10pt, letterpaper]{article}
\usepackage{fontspec} 

% DOCUMENT LAYOUT
\usepackage{geometry} 
%\geometry{a4paper, textwidth=5.5in, textheight=8.5in, marginparsep=7pt, marginparwidth=.6in}
\geometry{letterpaper, textwidth=5.5in, textheight=9.0in, marginparsep=7pt, marginparwidth=.6in}
\setlength\parindent{0in}

% FONTS
\usepackage{xunicode}
\usepackage{xltxtra}
\defaultfontfeatures{Mapping=tex-text} % converts LaTeX specials (``quotes'' --- dashes etc.) to unicode
\setromanfont [Ligatures={Common},Numbers={OldStyle}]{Adobe Caslon Pro}
\setmonofont[Scale=0.8]{Monaco} 
\setsansfont[Scale=0.9]{Optima Regular} 
% ---- CUSTOM AMPERSAND
\newcommand{\amper}{{\fontspec[Scale=.95]{Adobe Caslon Pro}\selectfont\itshape\&}}
% ---- MARGIN YEARS
\usepackage{marginnote}
\newcommand{\years}[1]{\marginnote{\scriptsize #1}}
\renewcommand*{\raggedleftmarginnote}{}
\setlength{\marginparsep}{7pt}
\reversemarginpar

% \usepackage[misc]{ifsym} %adds email symbol ('\Letter')
\usepackage{marvosym}  %adds cell phone symbol ('\Mobilefone')
\usepackage{changepage} %allows left/right indentation of body text

% HEADINGS
\usepackage{sectsty} 
\usepackage[normalem]{ulem} 
\sectionfont{\rmfamily\mdseries\LARGE} 
\subsectionfont{\rmfamily\mdseries\scshape\Large} 
\subsubsectionfont{\rmfamily\bfseries\upshape\normalsize} 
% \subsectionfont{\rmfamily\mdseries\scshape\normalsize} 
% \subsubsectionfont{\rmfamily\bfseries\upshape\normalsize} 

% JAPANESE LANGUAGE SETTINGS
\usepackage{xeCJK}
\setCJKmainfont{MS Mincho} % for \rmfamily
% \setCJKsansfont{MS Gothic} % for \sffamily

% PDF SETUP
% ---- FILL IN HERE THE DOC TITLE AND AUTHOR
\usepackage[xetex, bookmarks, colorlinks, breaklinks, pdftitle={john_ortiz_cv},pdfauthor={John P. Ortiz}]{hyperref}  
%\hypersetup{linkcolor=blue,citecolor=blue,filecolor=black,urlcolor=blue} 
%\hypersetup{linkcolor=black,citecolor=blue,filecolor=black,urlcolor=black} 
\hypersetup{linkcolor=black,citecolor=blue,filecolor=black,urlcolor=blue} 

% define a double line separator
\def\doubleline{

	\vspace{-1.4em}
	\hspace{\fill}\linethickness{0.7pt}\line(1,0){5.5in}\hspace{\fill}
	
	\vspace{-1.0em}
	\hspace{\fill}\linethickness{0.7pt}\line(1,0){5.5in}\hspace{\fill}
	
}

% CENSORING PRIVATE INFO
% use \censor{} to black out pieces of text
% use \StopCensoring to turn off all instances of censoring
\usepackage{censor}

% %  (FOR DISSERTATION): custom page number start
% \setcounter{page}{252}

% MULTIPLE COLUMNS
\usepackage{multicol}

% NICE 1st, 2nd, formatting 
\usepackage[super]{nth}

% Load commands with citation data output from ``get_scholar_stats.py`` script
% (citations.tex)
\newcommand{\citdate}{01 April, 2024}
\newcommand{\cittotal}{122}
\newcommand{\cithindex}{7}
\newcommand{\citiindex}{7}


%-----------------------------------------------------------------------------------

% DOCUMENT
\begin{document}
% {\LARGE John P. Ortiz} %\\[1cm]
{\Huge John P. Ortiz} %\\[1cm]
%\vfill

%% centered info with double line underneath
%\begin{center}
%	{\LARGE John P. Ortiz} \\[0.5cm]
%	799 6th St. \#14, Los Alamos, NM 87544 •\ (541) 207-5846 •\ john.p.ortiz.14@gmail.com
%\end{center}
%\doubleline

\StopCensoring     %<------------------ UNCOMMENT TO TURN OFF CENSORING
\begin{center}
	% \begin{tabular}{l l}
    %%--- tabular* sets the table width and expands by filling the separator
    \setlength\tabcolsep{0pt}
    \begin{tabular*}{\linewidth}{@{\extracolsep{\fill}} l l}
        Postdoctoral Research Associate   &
        \hspace{.5in}\href{https://lanlexperts.elsevierpure.com/en/persons/john-philip-ortiz}{LANL Experts Profile} \\ 
        National Security Earth Science Group (EES-17)   &
        % National Security Geoscience Group (EES-17) \ \ \ \ \ \ \  &
        \hspace{.5in}\href{https://johnportiz14.github.io}{johnportiz14.github.io}
        \\
        Earth \& Environmental Science Division         & \hspace{.5in}\Email {     }
        \href{mailto:jportiz@lanl.gov}{jportiz@lanl.gov}
        \\ 	
        Los Alamos National Laboratory &
        \hspace{.5in}\Mobilefone { } \censor{+1 (541) 207-5846}  \\
        % %
        % Ph.D. Candidate   &
        % \hspace{.5in}\href{https://lanlexperts.elsevierpure.com/en/persons/john-philip-ortiz}{LANL Experts Profile} \\ 
        % Johns Hopkins University   &
        % \hspace{.5in}\href{https://johnportiz14.github.io}{johnportiz14.github.io}
        % \\
        % Whiting School of Engineering       & \hspace{.5in}\Email {     }
        % \href{mailto:jportiz@lanl.gov}{jportiz@lanl.gov}
        % \\ 	
        % Department of Environmental Health \& Engineering  &
        % \hspace{.5in}\Mobilefone { } \censor{+1 (541) 207-5846}  \\
	\end{tabular*}
	% \end{tabular}
\end{center}

%%\hrule
%\section*{Current position}
%\emph{Emeritus Professor}, Institute for Advanced Study, Princeton

%-----------------------AREAS OF SPECIALIZATION-----------------------
%%\hrule
\section*{Areas of Specialization}
	% numerical modeling •\ hydrogeology •\ fractures •\
	% planetary science •\ petroleum geofluids
	% 
\begin{center}
    % numerical modeling •\ hydrogeology •\ fracture-matrix interactions •\
    multiphase flow \& transport •\ hydrogeology •\ fracture-matrix interactions •\ \\
    computational physics model development •\ reactive transport •\ planetary science %•\ petroleum geofluids
\end{center}

%--------------------------EDUCATION-------------------------
\section*{Education}
\noindent
\newlength\q
\setlength\q{\dimexpr .5\textwidth -2\tabcolsep}

\years{2024}\textsc{Johns Hopkins University (JHU)}

    \noindent
	% \begin{tabular}{p{\q}p{\q}} 
	% \textsc{Ph.D.} Candidate in Environmental Health \& Engineering & 
    % GPA: 3.53/4.00
    % \end{tabular}\\
    \textsc{Ph.D.} Candidate in Environmental Health \& Engineering \hfill GPA: 3.53/4.00\\
    \textsc{Dissertation:} Subsurface Flow and Transport Processes with
    Applications to Methane Variations on Mars\\
    \textsc{Advisor:} Dr. Harihar Rajaram \\

\years{2017}\textsc{New Mexico Institute of Mining and Technology (NMT)}

    \textsc{M.Sc.} in Hydrology \hfill GPA: 3.89/4.00\\
	\textsc{Thesis:} The role of fault-zone architectural elements and basal
	altered zones on downward pore pressure propagation and induced seismicity
	in the crystalline basement\\
    \textsc{Advisor:} Dr. Mark Person\\

\years{2015}\textsc{Dartmouth College}

    \textsc{B.A.} in Earth Sciences with Honors \hfill GPA: 3.47/4.00\\
    \textsc{Honors Thesis:} Quantifying regional sediment flux from
    observations of nearshore morphology in the Columbia River Littoral Cell\\
    \textsc{Advisor:} Dr. W. Brian Dade


%\years{2017}\textsc{M.Sc.} in Hydrology, New Mexico Institute of Mining and Technology (NMT)
%\begin{adjustwidth}{5pt}{0pt}
%
%	\textsc{Thesis:} The Role of Fault-Zone Architectural Elements and Basal Altered Zones on Downward Pore Pressure Propagation and Induced Seismicity in the Crystalline Basement
%%	\begin{tabular}{@{}l l}
%%	\textsc{Advisor:} Dr. Mark Person & \hspace{1cm} \textsc{GPA:} 3.89/4.00 
%	 \textsc{GPA:} 3.89/4.00
%%\end{tabular}\\
%\end{adjustwidth}
%
%
%\years{2014}\textsc{B.A.} in Earth Sciences, Dartmouth College\
%\begin{adjustwidth}{5pt}{0pt}
%	\textsc{Honor's Thesis:} Quantifying regional sediment flux from observations of nearshore morphology in the Columbia River Littoral Cell 
%\end{adjustwidth}

%--------------------------PLANETARY MISSION EXPERIENCE--------------
\section*{Planetary Exploration Mission Experience}
\label{sec:missions}  %label for linking to sections
\noindent
\years{2023}NASA TLS-SAM Experiment Proposal, Co-author

    Co-authored proposal with members of TLS-SAM (Tunable Laser Spectrometer
    within the Sample Analysis at Mars) team proposing strategic timing of
    atmospheric sample experiments for Mars Science Laboratory (MSL)
    \textit{Curiosity} rover to perform in order to constrain sub-diurnal
    methane variations at Gale crater, Mars. MSL \textit{Curiosity}
    successfully executed one of the proposed experiments 23 September, 2023,
    and a second on 10 December, 2023. Authors: Daniel Lo (Univ. of Michigan),
    Sushil Atreya (Univ. of Michigan), Scot Rafkin (Southwest Research
    Institute), Jorge Pla-Garc\'{i}a (Centro de Astrobiolog\'{i}a y Instituto Nacional de T\'{e}cnica Aeroespacial; CSIC-INTA),
    John Moores (York University), Daniel Vi\'{u}dez-Moreiras (CSIC-INTA), John
    Ortiz.

%--------------------------HONORS/AWARDS------------------------------
%\hrule
%\section*{Grants, honors \amper{} awards}
\section*{Honors \amper{} Awards}
\label{sec:awards}  %label for linking to sections
\noindent
%\begin{adjustwidth}{5pt}{0pt}
%\end{adjustwidth}
\years{2024}``Spot'' Performance Award (January), Los Alamos National Laboratory

    \emph{For outstanding performance and lasting contribution in support of the
    Laboratory's mission and values.}\\
    From the award:
    ``John's work on understanding methane transport on Mars, funded through an
    LDRD CSES Student Fellowship, has produced very high quality results
    gaining high visibility for LANL. His recent \textit{Journal of Geophysical
    Research: Planets} article was picked up by news servers around the world,
    including \textit{Newsweek} in the US. Finally, John's results were used to
    guide sample collection on Mars from the \textit{Curiosity} rover, building
    new ties between NASA and LANL.''\\

\years{2020}R\&D 100 Award Winner (\emph{Amanzi-ATS}), Contributor

    Contributed to the Amanzi-ATS multiphase flow and transport simulator
    project by augmenting code verification and benchmark tests in addition to
    maintaining user guide documentation. The code won an R\&D 100 Award in
    September 2020.
    \href{https://www.rdworldonline.com/rd-100-award-winners-announced-in-mechanical-materials-category/}{rdworldonline.com}\\

\years{2018}Top 20 Most Downloaded Recent Papers, Wiley Publishing (\emph{Geofluids})
% \begin{adjustwidth}{5pt}{0pt}

	Amongst articles published between July 2016 and June 2018, the article
	``Exploring the potential linkages..." (see
	\hyperref[sec:pubs]{Publications \& Presentations}) was in the top 20 for
	number of downloads in the 12-months post online publication.\\
% \end{adjustwidth}

\years{2018}“Spot" Performance Award (May), Los Alamos National Laboratory
% \begin{adjustwidth}{5pt}{0pt}

    Going above and beyond the call of duty under tight and/or emergency
    deadlines to finish a project deliverable in the form of a software package
    that rapidly predicts underground nuclear explosion (UNE) subsurface
    fractured-rock gas tracer transport arrival times and detection windows to
    the earth's surface. Software was delivered to AFTAC (Air Force Technical
    Applications Center).
% \end{adjustwidth}

%--------------------------RESEARCH GRANTS------------------------------
\section*{Research Funding}
\label{sec:funding}
\noindent
\emph{The asterisk (*) indicates a grant for which I was not an official PI due
to graduate student status, but on which I was the lead writer and researcher.}\\

\years{2021}\textsc{*CSES Student Fellowship, Planetary Science}

    In-depth 3-year investigation of subsurface methane transport in fractured
    rock environments on Mars. Multiple gas-phase methane transport and release
    mechanisms were be considered in addition to more accurate prediction of
    atmospheric methane concentrations through subsurface-atmosphere coupling
    of \textsc{FEHM} subsurface flow \& transport simulator and an atmospheric mixing model
    implemented in Python.
    PI: Phil Stauffer (LANL); Student Fellow: John Ortiz (JHU-LANL); Co-Is:
    Roger Wiens (LANL), Harihar Rajaram (JHU), Kevin Lewis (JHU), Anthony Toigo
    (JHU-APL). Total: \$190k (\$126k).\\

\years{2021}\textsc{New Mexico Small Business Assistance (NMSBA) Program}

    Provided client/small-business owner a detailed workflow for analyzing
    natural fluctuations in water levels in wells due to solid-earth body tides
    in order to determine aquifer properties. The workflow is covered in a
    $\sim$150-page user guide that provides detailed instructions for using
    several pre-existing, Open-source programs to perform the analysis. The
    workflow allows the client to quickly and cheaply calculate aquifer
    properties as an alternative to performing expensive well pump tests.
	PI: Phil Stauffer (LANL). Total: \$17k (\$15k). \\

\years{2020}\textsc{New Mexico Small Business Assistance (NMSBA) Program}

	Seed funding to pursue preliminary work in tidal analysis for a client who
	owns a small environmental consulting firm. By analyzing water levels in
	wells in response to earth tides, aquifer properties can be calculated
	without the need to perform expensive pump tests.  Investigated tidal
	analysis background and methods during the first year of funding and
	delivered a 20-page report to the client on the feasibility of this method,
	highlighting a path toward a workflow that could eventually be used by the
	client.   
	PI: Phil Stauffer (LANL). Total: \$17k (\$15k). \\

\years{2020}\textsc{*CSES Rapid Response Research \& Development Grant, Planetary
Science}

	Seed funding for preliminary work and proof-of-concept results of
	subsurface methane gas transport in fractured-rock environments on Mars
	influenced by periodic barometric pressure variations.
    PI: Dylan Harp (LANL); Student: John Ortiz (JHU-LANL); Co-Is: Roger Wiens
    (LANL), Harihar Rajaram (JHU), Kevin Lewis (JHU). Total: \$30k (\$24.7k).


%-----------------------RESEARCH EXPERIENCE---------------------------
\section*{Research Experience}
% \setlength{\parskip}{5pt}
\setlength{\parindent}{14pt}
\years{2020 - pres.}\textsc{Doctoral Student, GRA (Graduate Research Assistant)}\\
\textit{Energy and Natural Resources Security Group (EES-16), Los Alamos National Laboratory}

    \vspace{3pt} \noindent  
    Wrote two successfully-funded LDRD CSES Planetary Science grants (see
    \hyperref[sec:funding]{Research Funding}) to pursue work modeling
    subsurface methane transport in fractured rock at Gale crater, Mars at 70\%
    FTE for three years. Key outcomes were two published first-authored journal
    articles and co-authorship of a NASA proposal with the TLS-SAM Team for
    multiple atmospheric sampling experiments to be performed by the MSL
    \textit{Curiosity} rover, forging new ties between NASA and LANL (see
    \hyperref[sec:missions]{Planetary Exploration Mission Experience} and
    \hyperref[sec:funding]{Research Funding}). Additional outcomes include release
    of the Mars-specific branch of FEHM (\textsc{FEHM-Mars};
    \href{https://doi.org/10.5281/zenodo.10455952}{https://doi.org/10.5281/zenodo.10455952})
    that feature code modifications in Fortran to adapt to martian conditions (e.g., reduced
        gravity, equation-of-state modifications consistent with Mars atmospheric
    ``air'') and a reactive transport capability allowing users to specify
    time-varying distribution coefficients for tracer adsorption problems related
    to temperature changes.

    Collaborated with experimentalists in EES-14 and EES-16 to develop reactive
    gas transport models to determine transport properties of noble gases in
    variably saturated zeolitic and non-zeolitic rocks based on novel data from
    bench-scale diffusion experiments. Augmented reactive transport
    capabilities of FEHM by developing a new dual-site, competitive kinetic
    adsorption model to interrogate counter-intuitive radionuclide gas
    transport experimental results in zeolitic rocks. This work will improve
    gas tracer signal interpretation related to historical UNE test data and
    inform field-scale transport models.

    Carried out programmatic work for government sponsors related to detecting
    and verifying underground nuclear explosion (UNE) tests. Performed pressure
    and gas transport predictions to support Low Yield Nuclear Monitoring
    (LYNM) program.  Was also the lead developer Numerical Reduced Order
    Multiphase Model (NROMM) software package for rapid prediction of gas
    seepage times, which was developed for AFTAC (Air Force Technical
    Applications Center) as part of a Trailblazer project (see
    \hyperref[sec:awards]{Honors \& Awards}). The NROMM tool is a Python
    wrapper for FEHM that allows the user to use a simple, readable input file
    to run flow \& transport UNE simulations with reactive tracers while
    handling meshing, initialization calculations, radioactive decay, and
    post-processing. Led a training workshop for end users of NROMM to members of
    the \nth{21} Surveillance Squadron of AFTAC  (see
    \hyperref[sec:workshopsFacilitated]{Workshops Facilitated}). 

	Led a NMSBA (New Mexico Small Business Assistance;
	\href{https://www.nmsbaprogram.org/}{https://www.nmsbaprogram.org/})
	project (2021-24) to help a small environmental consulting firm dramatically reduce
	costs associated with interpreting aquifer hydrogeologic properties from
	well data.  Demonstrated the use of frequency domain analysis and
	solid-earth tidal loading to estimate hydrogeologic properties without the
	need for expensive pump tests. \\

\noindent
\years{2017 - 2019}\textsc{Post-Master's Student, GRA}\\
\textit{Computational Earth Science Group (EES-16), Los Alamos National Laboratory}

    \vspace{3pt} \noindent  
	Lead developer of the Numerical Reduced Order Multiphase Model (NROMM)
	software package for rapid prediction of gas seepage times. Software
	application was developed for use on multiple platforms
	(Windows/Linux/MacOS) and was delivered to AFTAC.

	Developed numerical approaches for detecting and verifying underground
	nuclear explosion (UNE) tests. Projects included simulating
	radionuclide gas transport in fractured geologic media using finite-element
	method (FEM) and control volume finite-element (CVFEM) numerical models,
	simulating high-pressure methane injection into shale samples to inform
	laboratory investigations, and determining laboratory- and field-scale
	transport properties of rocks using models and tracer experiments.

    Developed a workflow for coupling discrete fracture networks (DFNs)
    generated in \textsc{dfnWorks} with a 3-D continuum mesh as a means of
    improving computational efficiency of flow and transport simulations
    related to UNEs.  Developed and performed a suite of flow and transport
    verification tests using FEHM. Such model frameworks are useful for
    representing scenarios where the 3-D rock matrix or spherical cavity must
    be explicitly represented and coupled to planar fracture features.

    Contributed to development of the Amanzi-ATS high performance computing
    (HPC) flow \& transport simulator to meet the Nuclear Quality Assurance-1
    (NQA-1) regulatory standard by improving code verification and benchmark
    tests in addition to maintaining software user guide documentation.
    Collaborated on a multi-lab program supported by the DOE Office of
    Environmental Management (DOE EM) to provide scientifically defensible and
    standardized assessments of the uncertainties and risks associated with the
    environmental cleanup and closure of waste sites. This software won an R\&D
    100 Award in September 2020 (see \hyperref[sec:awards]{Honors \&
    Awards}).\\

\noindent
\years{2016 - 2017}\textsc{Graduate Research Assistant}\\
\textit{Earth and Environmental Sciences Department, NMT}

    \vspace{3pt} \noindent  
    Created transient 3-D finite-difference (FDM) models in MODFLOW to analyze
    induced seismicity risk resulting from fluid-fault interactions associated
    with a suite of basal reservoir injection scenarios. Also developed
    transient 2-D cross-sectional FDM models in MATLAB to test fluid-fault
    interactions for crystalline basement fault zones exhibiting local,
    dynamically enhanced permeability caused by excess fluid pressures. 
    Developed new approaches for representing fault zones with multiple
    architectural components and identified several key hydrogeologic
    parameters that control deep propagation of the fluid pressure envelope and
    thus present increased risk of induced or triggered seismic events.

	Collaborated on an NSF-funded (via NM EPSCoR) field project deploying
	subsurface geophysical field survey equipment (transverse electromagnetics [TEM],
	magnetotellurics [MT]) for interpretation of deep saline geothermal flow
	regimes in order to evaluate potential hydrothermal systems in southern New
	Mexico.\\

\noindent
\years{2013}\textsc{REU Research Intern}\\
\textit{College of Earth, Ocean, and Atmospheric Sciences, Oregon State University}

    \vspace{3pt} \noindent  
    Completed an NSF-funded Research Experience for Undergraduates (REU)
    internship on coastal processes. Collected nearshore single-beam bathymetry
    data using jet skis. Interpolated nearshore topographic and bathymetric
    data to extract key spatial and temporal morphology metrics using MATLAB.
    Determined rates of longshore-uniform sandbar migration cycles along the
    Oregon and Washington coasts representing a huge component of seasonal
    coastal sediment flux.\\

\noindent
\years{2013}\textsc{Research Intern}\\
\textit{United States Army Corps of Engineers Field Research Facility, Duck NC}

    \vspace{3pt} \noindent  
    Supported active coastal evolution and erosion research by collecting and
    processing LiDAR observations during and after storm surges using both
    beach-based and amphibious vehicles (LARC-V; Lighter Amphibious Resupply
    Cargo 5-ton).

	Created a paleo-hurricane record of St. Croix by performing grain-size
	analysis on sediment cores combined with charcoal carbon dating.

%-----------------------WORKSHOPS FACILITATED-------------------------
\setlength{\parindent}{0pt}  %<--- go back to zero paragraph indent

\section*{Workshops \amper{} Shortcourses}

\subsection*{Workshops Facilitated}
\label{sec:workshopsFacilitated}
\noindent
\years{2021}\textsc{NROMM Demo and Training for AFTAC}%\\

    Led a training workshop and software demonstration for members of the AFTAC
    \nth{21} Surveillance Squadron on use of the NROMM software package (see
    \hyperref[sec:awards]{Honors \& Awards}).  The NROMM tool I developed is a
    Python wrapper for FEHM that allows the user to use a simple, readable
    input file to run flow \& transport UNE simulations with reactive tracers
    while handling meshing, initialization calculations, radioactive decay, and
    post-processing.

\subsection*{Relevant Shortcourses}
\label{sec:shortcourses}
\years{2023} \textsc{LYNM Underground Nuclear Weapons Testing Orientation Program (UNWTOP)}\\
\textit{Nevada National Security Site (NNSS), Las Vegas, NV}

    Attended 4-day multi-lab program at the NNSS for nuclear intelligence
    orientation regarding history, challenges, and scientific topics related to
    underground nuclear weapons testing. 


%-----------------------TEACHING EXPERIENCE---------------------------
\section*{Teaching Experience}
\noindent
\years{2019}\textsc{Grading Assistant}\\
\textit{Environmental Health \& Engineering Department, Johns Hopkins University}

	Graded weekly homework assignments for 20+ upper-level undergraduate
	engineering students in an Introduction to Fluid Mechanics course.
	Occasionally taught course lectures as needed.\\

\years{2015 - 2016}\textsc{Graduate Teaching Assistant}\\
\textit{Earth and Environmental Sciences Department, NMT}

    Developed lesson plans and led class lectures and field trips for 20+
    students in intermediate and upper level undergraduate Earth Sciences lab
    sections. Taught lab sections for courses in both Geomorphology and Stratigraphy
    \& Paleontology with work load totaling 20 hrs/week.  

%--------------------------SKILLS-------------------------------
\section*{Skills}
\setlength{\parskip}{5pt}
\subsection*{Programming/command languages}
\noindent
\textbf{\textsc{Python}} -- Advanced proficiency. Designed and executed
sensitivity analyses and parameter optimization algorithms for calibration of transport
parameters in FEHM and PFLOTRAN numerical simulations. Developed a workflow for
generating stochastic discrete fracture networks (DFNs) embedded in rock matrix
via upscaling of properties to given mesh dimensions, compatible with FEHM and
PFLOTRAN mesh and material files. Developed an atmospheric mixing model
governed by eddy diffusivity within the planetary boundary layer (PBL) using a
backward Euler finite-difference method scheme and coupled it to gas tracer
fluxes produced by FEHM simulations. Developed a standalone software
application for real-time predictions of gas breakthrough for use on Windows,
Linux, and MacOS platforms.

\textbf{\textsc{Fortran}} -- Intermediate proficiency in using the Fortran
language to augment current and legacy flow and transport codes (e.g.,
\textsc{FEHM}). Added reactive transport capability allowing for competitive
kinetic adsorption between multiple tracer species to multiple types of
adsorbent loading sites.  Developer and maintainer of a Mars-specific branch of
\textsc{FEHM} (\textsc{FEHM-Mars};
\href{https://doi.org/10.5281/zenodo.10455952}{https://doi.org/10.5281/zenodo.10455952})
that features code modifications to adapt to martian conditions (e.g., reduced
gravity, equation-of-state modifications for Mars ``air'') and a capability
allowing users to specify time-varying distribution coefficients for tracer
adsorption problems related to temperature changes.

\textbf{\textsc{Matlab}} -- Advanced proficiency in writing scripts for data
processing and developing a variety of iterative/numerical models to solve
hydrogeologic environmental problems.

\textbf{\textsc{R}} -- Intermediate proficiency in building a variety of
statistical and machine learning models to interpret and synthesize hydrologic,
geochemical, climatic, and environmental data. Collaborated on statewide New
Mexico water budget research project predicting reference evapotranspiration
using different statistical models on a wide range remote sensing parameters.

\textbf{\LaTeX} -- Advanced proficiency in typesetting/document preparation of
technical and scientific documentation.

\textbf{\textsc{Bash/shell}} -- Advanced proficiency in managing independent
code development environments and automating/batching tasks using command
line prompts and shell scripts. 


\subsection*{Software, codes}
\noindent
\textbf{\textsc{Fehm}} -- Expert proficiency in creating multiphase fluid
flow simulations with mass transport. Created test suites comparing gas
tracer transport within a fracture with matrix diffusion to an analytical
solution governed by a modified Richard's equation for unsaturated flow (with
immobile liquid phase) in order to test when multiphase flow significantly
affects gas tracer transport.
\href{https://github.com/lanl/FEHM}{github.com/lanl/FEHM}. Maintainer of
Mars-specific release (\textsc{FEHM-Mars};
\href{https://doi.org/10.5281/zenodo.10455952}{https://doi.org/10.5281/zenodo.10455952}).

\textbf{\textsc{Pflotran}} -- Intermediate proficiency in creating multiphase
fluid flow simulations with mass transport. Created test suites in similar
capacity to work done with FEHM. Also used in conjunction with FEHM to validate
the capabilities of DFNs to simulate
flow \& transport in fractured media under transient conditions.
\href{https://www.pflotran.org}{www.pflotran.org} 

\textbf{\textsc{LaGriT}} -- Intermediate proficiency in generating numerical
finite-element meshes for a variety of geological applications, including
porous flow and transport modeling and DFNs. Proficient at generating meshes
using the 3 interfaces: command line, batch driven via control file, and
PyLaGriT Python command/batch interface. Developed Python/\textsc{LaGriT}
workflow for translating explosive rock damage models to continuum
permeability-damage fields in geologic frameworks. Have also worked on
capability development by using novel approaches to combine continuum 3D grids
with 2D DFN planes into unified meshes for use in transient flow and transport
models.
\href{https://lagrit.lanl.gov/index.shtml}{lagrit.lanl.gov/index.shtml}

\textbf{\textsc{dfnWorks}} -- Intermediate proficiency in generating 3D
DFNs for simulating flow and transport, including
generating continuum representations of fractured porous media with octree
refinement using an upscaled discrete fracture matrix model (UDFM) workflow.
\href{https://dfnworks.lanl.gov}{dfnworks.lanl.gov}

\textbf{\textsc{Git}} -- Advanced proficiency in source code management/version control
in application development and collaborating with colleagues on coding and software
projects. Have experience maintaining Git repositories as a lead developer as
well as in a supporting collaborator role.

\textbf{\textsc{ParaView}} -- Advanced proficiency in interactive scientific
data visualization and exploration by generating static figures and
animations/videos. \href{https://www.paraview.org/}{www.paraview.org} 

\textbf{\textsc{GDB (GNU Debugger)}} -- Intermediate proficiency with command
line debugging tool for augmenting current and legacy flow and transport codes
written in Fortran and C++.

\textbf{\textsc{Modflow}} -- Advanced proficiency in developing 3D transient
and steady-state FDM models to solve a variety of hydrogeologic problems, 
including fluid-fault interactions as related to induced seismicity in thesis
work, and basin-scale backwards modeling of groundwater solute transport as
related to the Kirtland Air Force Base spill in Albuquerque, NM.

\textbf{\textsc{OpenFOAM}} -- Intermediate proficiency in developing
computational fluid dynamics (CFD) simulations of single and multiphase flow
and transport simulations in laminar and turbulent regimes.

\textbf{\textsc{SPOTL: Some Programs for Ocean-Tide Loading}} -- Intermediate
proficiency in computing load tides and strains produced on the solid earth by
both ocean tides and solid-earth body tides.
% \url{https://igppweb.ucsd.edu/~agnew/Spotl/spotlmain.html}
\href{https://igppweb.ucsd.edu/~agnew/Spotl/spotlmain.html}{https://igppweb.ucsd.edu/\(\sim \)agnew/Spotl/spotlmain.html}

\textbf{\textsc{Sphinx}} -- Advanced proficiency in generating and maintaining
professional documentation of software projects as HTML and \LaTeX \ output
from reStructuredText sources. Have supported documentation needs of various
collaborative code projects %(e.g., Amanzi-ATS, NROMM, SlugTide-octave). 
(e.g., \href{https://github.com/amanzi/amanzi}{Amanzi-ATS},
\href{https://gitlab.lanl.gov/jportiz/nromm_user_guide}{NROMM}, and
\href{https://github.com/johnportiz14/SlugTide-octave}{SlugTide-octave}).

\textbf{\textsc{Adobe Illustrator}} -- Advanced proficiency in creating and editing
publication-ready figures and illustrative schematic diagrams and animations for use in peer-reviewed journal articles and conference presentations.

\textbf{\textsc{Microsoft Excel}} -- Ranked \#7 in the world at the 2008 Microsoft Office
Worldwide Competition in Hawaii of 56,000 initial entrants
(www.betheworldchamp.com), Ireland National Champion (of 436 entrants).

% \textsc{COMSOL Multiphysics} – Beginner proficiency in modeling contaminant
% transport in academic coursework.\\[5pt] 
% 
% \textbf{\textsc{Petromod}} -- Intermediate proficiency in creating basin evolution,
% subsidence rate, and heat flow models of petroleum-bearing
% locales.
% 
% \textbf{\textsc{ArcGIS}} -- Intermediate proficiency in using digital elevation models (DEM)
% to delineate watersheds and create estimates of stream discharge in mountainous
% terrain.


\subsection*{Other}
\noindent
Advanced proficiency in written and conversational Spanish (nine years of
academic study).


%--------------------------PUBLICATIONS--------------------------
\section*{Publications \amper{} Presentations}
\label{sec:pubs}  %label for linking to sections
\setlength{\parskip}{0pt}
% h-index: 7 (as of February 2024)
% h-index: 7 (as of \today) 
h-index: \cithindex \ (as of \citdate)\\   % Read from citations.tex file
Total Citations: \cittotal\\
\nth{1}-author peer-reviewed journal articles: 3
% 3 First-author peer-reviewed journal articles

% \subsection*{Preprints}
% \noindent
% \years{2023}\textbf{Ortiz, J. P.}, Rajaram, H., Stauffer, P. S., 
% Lewis, K. W, Wiens, R. C., Harp, D. R. 
% Sub-diurnal methane variations on Mars driven by barometric pumping and
% planetary boundary layer evolution. 
% \emph{Authorea}. August 09, 2023. doi:10.22541/essoar.169160648.81638425/v1 \\
% \emph{(Submitted to JGR: Planets)}.

%\subsection*{Journal articles \amper{} Reports}
\subsection*{Journal articles \& Reports}

\subsubsection*{Peer reviewed}

\noindent
\hypersetup{linkcolor=black,citecolor=blue,filecolor=black,urlcolor=black} 
\years{2024}\textbf{Ortiz, J. P.}, Rajaram, H., Stauffer, P. S., Lewis, K. W,
Wiens, R. C., Harp, D. R.  Sub-diurnal methane variations on Mars driven by
barometric pumping and planetary boundary layer evolution. (2024). \emph{Journal of
Geophysical Research: Planets}. 129, e2023JE008043. \\
\href{https://agupubs.onlinelibrary.wiley.com/doi/epdf/10.1029/2023JE008043}{doi:10.1029/2023JE008043}.
\JournalofGeophysicalOOOOSubDiurnalMethaneVar \\ 
%
\years{2022}\textbf{Ortiz, J. P.}, Rajaram, H., Stauffer, P. S., Harp, D. R.,
Wiens, R. C., Lewis, K. W.  Barometric pumping through fractured rock: a
mechanism for venting deep methane to Mars' atmosphere. (2022). \emph{Geophysical
Research Letters}. 
\href{https://agupubs.onlinelibrary.wiley.com/doi/epdf/10.1029/2022GL098946}{doi:10.1029/2022GL098946}.
\GeophysicalResearchLOOOOBarometricpumpingthr\\
\hypersetup{linkcolor=black,citecolor=blue,filecolor=black,urlcolor=blue} 
(\underline{Cover article}:
\href{https://agupubs.onlinelibrary.wiley.com/doi/epdf/10.1002/grl.62460}{https://agupubs.onlinelibrary.wiley.com/doi/epdf/10.1002/grl.62460}).\\ 
%
\hypersetup{linkcolor=black,citecolor=blue,filecolor=black,urlcolor=black} 
\years{2022}Neil, C. W., Boukhalfa, H., Xu, H., Ware, S. D., \textbf{Ortiz, J.
P.}, Avendaño, S. T., Harp, D. R., Broome, S., Hjelm, R. P., Roback, R., Brug,
W. P., Stauffer, P. H. Gas diffusion through variably-water-saturated zeolitic
tuff: implications for transport following a subsurface nuclear event. (2022).
\emph{Journal of Environmental Radioactivity}.
\href{https://www.sciencedirect.com/science/article/pii/S0265931X22000959/pdfft?md5=9a2aafb80eab7a1daa481601c5af9247&pid=1-s2.0-S0265931X22000959-main.pdf}{doi:10.1016/j.jenvrad.2022.106905}.
\JournalofEnvironmentOOOOGasdiffusionthroughv\\
%
\years{2021}Avendaño, S. T., Harp, D. R., Kurwadkar, S., \textbf{Ortiz, J.
P.}, Stauffer, P. H. Continental-scale geographic trends in barometric-pumping
efficiency potential: a North American case study. (2021). 
\emph{Geophysical Research Letters}.  
\href{https://agupubs.onlinelibrary.wiley.com/doi/pdf/10.1029/2021GL093875}{doi:10.1029/2021GL093875}.
\GeophysicalResearchLOOOOContinentalScaleGeog\\
%
\years{2020}Petrie, E. S., Bradbury, K. K., Cuccio, L., Smith, K., Evans, J.
P., \textbf{Ortiz, J. P.}, Kerner, K., Person, M. A., Mozley, P. S.
Geologic characterization of nonconformities using outcrop and whole-rock core
analogues: hydrologic implications for injection-induced seismicity. (2020).
\emph{Solid Earth}, 11(5),1803-1821. 
\href{https://se.copernicus.org/articles/11/1803/2020/se-11-1803-2020.pdf}{doi:10.5194/se-11-1803-2020}.
% \SolidEarthDiscussionOOOOGeologiccharacteriza\\
\SolidEarthOOOOGeologiccharacteriza\\
%
\years{2020}Bourret, S. M., Kwicklis, E. M., Harp, D. R., \textbf{Ortiz, J.
P.}, Stauffer, P. H. Beyond Barnwell: Applying lessons learned from the
Barnwell site to other historic underground nuclear tests at Pahute Mesa to
understand radioactive gas-seepage observations. (2020). \emph{Journal of Environmental
Radioactivity}, 222. 
\href{https://www.sciencedirect.com/science/article/pii/S0265931X20300199/pdfft?md5=63166f84895846446b3a9ce93cc68c79&pid=1-s2.0-S0265931X20300199-main.pdf}{doi:10.1016/j.jenvrad.2020.106297}.
\JournalofEnvironmentOOOOBeyondBarnwellApplyi\\
%
\years{2019}Harp, D. R., \textbf{Ortiz, J. P.}, Stauffer, P. H. Identification
of dominant gas transport frequencies during barometric pumping of fractured
rock. (2019). \emph{Scientific Reports (Nature Publishing Group)}, 9(1), 9537.
\href{https://www.nature.com/articles/s41598-019-46023-z.pdf}{doi:10.1038/s41598-019-46023-z}.
\ScientificReportsOOOOIdentificationofdomi\\ 
%
\years{2019}\textbf{Ortiz, J. P.}, Person,
M. A., Mozley, P. S., Evans, J. P., Bilek, S. L. The role of fault-zone
architectural elements on pore pressure propagation and induced seismicity.
(2019). \emph{Groundwater}, 57(3): 465-478. 
\href{https://ngwa.onlinelibrary.wiley.com/doi/epdf/10.1111/gwat.12818}{doi:10.1111/gwat.12818}.
\GroundwaterOOOOTheRoleofFaultZoneAr\\
%
\years{2019}Stauffer, P. H., Rahn, T., \textbf{Ortiz, J. P.}, Salazar, L. J.,
Boukhalfa, H., Behar, H. R., Snyder, E. E. Evidence for High Rates of Gas
Transport in the Deep Subsurface. (2019). \emph{Geophysical Research Letters}.
\href{https://agupubs.onlinelibrary.wiley.com/doi/epdf/10.1029/2019GL082394}{doi:10.1029/2019GL082394}.
\GeophysicalResearchLOOOOEvidenceforHighRates\\ 
%
\years{2018}Harp, D. R., \textbf{Ortiz, J. P.},
Pandey, S., Karra, S., Anderson, D., Bradley, C., Viswanathan, H., \& Stauffer,
P. H. Immobile Pore-Water Storage Enhancement and Retardation of Gas Transport
in Fractured Rock. (2018). \emph{Transport in Porous Media}, 1-26.
\href{https://link.springer.com/content/pdf/10.1007/s11242-018-1072-8.pdf}{doi:10.1007/s11242-018-1072-8}.
\TransportinPorousMedOOOOImmobileporewatersto\\ 
%
\years{2017}\textbf{Ortiz, J. P.} The Role of Fault-Zone Architectural Elements
and Basal Altered Zones on Downward Pore Pressure Propagation and Induced
Seismicity in the Crystalline Basement. (2017). [Master's Thesis] \emph{New
Mexico Institute of Mining and Technology.}
\NewMexicoInstituteofOOOOTheRoleofFaultZoneAr\\ 
%
\years{2016}Zhang, Y., Edel, S. S., Pepin, J., Person, M., Broadhead, R.,
\textbf{Ortiz, J. P.}, Bilek, S. L., Mozley, P. S., \& Evans, J.  P. Exploring
the potential linkages between oil-field brine reinjection, crystalline
basement permeability, and triggered seismicity for the Dagger Draw Oil field,
southeastern New Mexico, USA, using hydrologic modeling. (2016). \emph{Geofluids}.
\href{https://onlinelibrary.wiley.com/doi/epdf/10.1111/gfl.12199}{doi:10.1111/gfl.12199}
\GeofluidsOOOOExploringthepotentia\\
%
\years{2014}\textbf{Ortiz, J. P.} Quantifying regional sediment flux from
observations of nearshore morphology in the Columbia River Littoral Cell.
(2014). [Undergraduate Thesis] \emph{Dartmouth College Senior Honors Thesis
Collection.}\\ 
%
\years{2014}Cohn, N., Ruggiero, P., \textbf{Ortiz, J. P.}, \& Walstra, D. J.
Investigating the role of complex sandbar morphology on nearshore
hydrodynamics. (2014). \emph{Journal of Coastal Research}, 70(sp1), 53-58.
\href{https://bioone.org/journalArticle/Download?urlId=10.2112%2FSI65-010.1}{doi:10.2112/SI65-010.1}.
\JournalofCoastalReseOOOOInvestigatingtherole

\subsubsection*{Not peer reviewed}
%
\years{2018}Stauffer, P. H., Rahn, T. A., \textbf{Ortiz, J. P.}, Salazar, L.
J., Boukhalfa, H., \& Snyder, E. E. Summary of a Gas Transport Tracer Test in
the Deep Cerros Del Rio Basalts, Mesita del Buey, Los Alamos NM. United States.
(2018). \href{https://www.osti.gov/servlets/purl/1417180}{doi:10.2172/1417180}.
\LosAlamosNationalLabOOOOSummaryofaGasTranspo\\
%

% Make links blue again
\hypersetup{linkcolor=black,citecolor=blue,filecolor=black,urlcolor=blue} 


\subsection*{Invited Talks}
\label{sec:invitedTalks}
\years{2023} ``The Mars Underground: Characterizing subsurface methane seepage
on the Red Planet'', Los Alamos Geological Society monthly meeting. 18 April,
2023.\\ 
%
\years{2023} ``The Mars Underground: Characterizing subsurface methane seepage
on the Red Planet'', New Mexico Bureau of Geology Seminar Series, New
Mexico Bureau of Geology and Mineralogical Resources. 7 April, 2023.\\
%



%\subsection*{Conference Talks \amper{} Posters}
%\subsection*{Conference Talks \& Posters}

\subsection*{Conference \& Meetings Talks}
\years{2024} ``Preferential adsorption of xenon in variably-saturated zeolitic tuff
'', LYNM Quad-Laboratory All-Hands Meeting, Lawrence Livermore National Laboratory.
25 June, 2024.\\
%
\years{2023} 
% \textbf{Ortiz, J. P.}, Rajaram, H., Stauffer, P. S., Lewis, K. W, Wiens, R. C.,
% Harp, D. R. 
``Sub-diurnal methane variations on Mars driven by barometric pumping and
planetary boundary layer evolution''. \textit{Session: P44C The New Mars
Underground III Oral}. AGU 2023 Fall Meeting in San Francisco, CA. 14 December,
2023.\\
%
\years{2023} ``The Mars Underground: Characterizing methane seepage on the Red
Planet'', LANL Center for Space and Earth Science (CSES) Symposium. 23 August,
2023.\\ 
%
\years{2023} ``The Mars Underground: Characterizing methane seepage on the Red
Planet'', LDRD Appraisal for FY21-23, CSES Planetary Science Student Fellow. 17 April,
2023.\\
%
\years{2023} ``NROMM: Numerical Reduced-Order Multiphase Model | A tool for
making rapid predictions of UNE gas seepage'', LYNM Program Technical Meeting,
Los Alamos National Laboratory. 23 January, 2023.\\
%
\years{2022} ``Using legacy radionuclide data to validate barometric pumping
models'', LYNM Quad-Laboratory All-Hands Meeting, Sandia National Laboratory.
25 May, 2022.\\
%
\years{2022} ``PE1-A pressure and permeability predictions'', LYNM
Quad-Laboratory All-Hands Meeting, Sandia National Laboratory. 26 May, 2022.\\
%
\years{2021}
``PE1-A pressure and gas arrival predictions based on in-situ permeability
measurements'', Nuclear Test Monitoring Exchange of Information by Visit and
Report (EIVR) 58. 12 October, 2021.\\
%
\years{2018}
% \textbf{Ortiz, J. P.}, Harp, D. R., Stauffer, P. H. 
``A reduced-order model to assist real-time predictions of gas transport in
unsaturated fractured media''. InterPore 10\textsuperscript{th} Annual Meeting in
New Orleans, LA. May 2018.\\
%
\years{2018}
% Harp, D. R., \textbf{Ortiz, J. P.}, Stauffer, P. H., Viswanathan, H., Anderson,
% D. N., Bradley, C. 
``Analysis of enhanced gas transport in fractured rock due to barometric pressure
variations'' (presenting for Dylan Harp). InterPore 10\textsuperscript{th} Annual
Meeting in New Orleans, LA. May 2018.\\
%
\years{2017}
% \textbf{Ortiz, J. P.}, Ortega, A. D., Harp, D. R., Boukhalfa, H.
``Improving estimates of subsurface gas transport in unsaturated fractured media
using field tracer data and numerical methods.''  \textit{Session: H52A Advances
    in Hydrological Characterization of Flow and Transport in Fractured Media:
Numerical and Experimental Observations II}. AGU 2017 Fall Meeting in New
Orleans, LA. 15 December 2017. \\
%
\years{2016}
% \textbf{Ortiz, J. P.}, Person, M. A., Mozley, P. S., Evans, J. P.
``The Hydrologic Connection Between Basal Reservoir Injection, Crystalline
Basement Fault Zones, and Induced Seismicity''.  GSA Annual Meeting in Denver,
CO. September 2016.\\


\subsection*{Conference Posters}
\years{2023}\textbf{Ortiz, J. P.}, Rajaram, H., Stauffer, P. S., Lewis, K. W,
Wiens, R. C., Harp, D. R. Sub-diurnal methane variations on Mars driven by
barometric pumping and planetary boundary layer evolution. \textit{LANL Student
Symposium}.\\
%
\years{2022}\textbf{Ortiz, J. P.}, Rajaram, H., Stauffer, P. H., Harp, D. R.,
Wiens, R. C., Lewis, K. W. Barometric pumping through fractured rock: A
mechanism for venting deep underground methane to Mars' atmosphere.
\textit{Session: P22F The New Mars Underground: Nexus of Decadal Planetary
Science Objectives II Poster}. AGU 2022 Fall Meeting in Chicago, IL.
\AGUFallMeetingAbstraOOOOBarometricPumpingThr\\ 
%
\years{2018}\textbf{Ortiz, J. P.}, Harp, D. R., Stauffer, P. H., Gable, C. W.,
Makedonska, N. Coupled discrete fracture and 3D continuum domain representation
to efficiently capture gas transport from underground cavities.
\textit{Session: H51P Coupled Processes in Fractured Media Across Scales:
Experimental and Modeling Advances Posters}. AGU 2018 Fall Meeting in
Washington D.C. 
% \AGUFallMeetingAbstraOOOOCoupleddiscretefract\\
%
\years{2018}Harp, D. R., \textbf{Ortiz, J. P.}, Kwicklis, E. M., Bourret, S.
M., Viswanathan, H., Stauffer, P. H. Identifying dominant barometric
frequencies driving gas transport in fractured rock (presenting for Dylan
Harp). \textit{Session: H51P Coupled Processes in Fractured Media Across
Scales: Experimental and Modeling Advances Posters}. AGU 2018 Fall Meeting in
Washington D.C. 
\AGUFallMeetingOOOOIdentifyingdominantb\\
%
\years{2013}\textbf{Ortiz, J. P.}, Ruggiero, P., Cohn, N. Interannual Sandbar
Variability within the Columbia River Littoral Cell.
\textit{Session: EP13A Coastal Geomorphology and Morphodynamics I Posters}.
AGU 2013 Fall Meeting in San Francisco, CA.


\subsection*{News Coverage/Web Articles}
% \subsection*{News/General Interest Articles}
\noindent
\years{2024}``Mystery of Mars' `Burps' Could Aid Search for Life''.
\emph{\textbf{Newsweek}}, 25 January 2024.\\
% \emph{Newsweek}, 25 January 2024.\\
\href{https://www.newsweek.com/mystery-mars-burp-belch-methane-search-life-1863907}
{https://www.newsweek.com/mystery-mars-burp-belch-methane-search-life-1863907}.\\
% 
\years{2024}{``Study Predicts Best Times for Rover to Sample Mars Methane in
Search for Life''. 
\emph{Johns Hopkins University Engineering News}, 26 January 2024.
\href{https://engineering.jhu.edu/news/study-predicts-best-times-for-rover-to-sample-mars-methane-in-search-for-life/
}{https://engineering.jhu.edu/news/study-predicts-best-times-for-rover-to-sample-mars-methane-in-search-for-life/}.\\ 
%
\years{2024}{``Atmospheric Pressure Changes Could Be Driving Mars' Elusive
Methane Pulses''}. 
\emph{Los Alamos Daily Post}, 26 January 2024.
\href{https://ladailypost.com/lanl-atmospheric-pressure-changes-could-be-driving-mars-elusive-methane-pulses/}{https://ladailypost.com/lanl-atmospheric-pressure-changes-could-be-driving-mars-elusive-methane-pulses/}.\\
%
\years{2024}``Methane pulses on Mars possibly driven by atmospheric pressure
changes''. 
\emph{Phys.org}, 24 January 2024.  
\href{https://phys.org/news/2024-01-methane-pulses-mars-possibly-driven.html}{https://phys.org/news/2024-01-methane-pulses-mars-possibly-driven.html}.\\
%
\years{2024}``Mars' methane release tied to atmospheric pressure fluctuations''.
\emph{New Mexico Sun}, 25 January 2024.
\href{https://newmexicosun.com/stories/653884854-mars-methane-release-tied-to-atmospheric-pressure-fluctuations}{https://newmexicosun.com/stories/653884854-mars-methane-release-tied-to-atmospheric-pressure-fluctuations}.\\
%
\years{2024}``Mars' methane release tied to atmospheric pressure fluctuations''.
\emph{T{\"u}rkiye Newspaper}, 25 January 2024.
\href{https://www.turkiyenewspaper.com/science/17875}{https://www.turkiyenewspaper.com/science/17875}.\\
% 
\years{2024}{``火星でメタンを測るには日の出の直前が最適?メタン濃度の変化をモデル計算で予測 
    [Is the best time to measure methane on Mars just before sunrise?
    Predicting changes in methane concentration using model calculations]''. 
\emph{Sorae}, 12 February 2024.
\href{https://sorae.info/astronomy/20240212-mars-methane.html}{https://sorae.info/astronomy/20240212-mars-methane.html}\\
%
\years{2024}``Atmospheric pressure changes could be driving Mars' elusive
methane pulses''.  \emph{LANL News Stories}, 24 January 2024.
\href{https://discover.lanl.gov/news/0124-mars-methane-pulses/}
{https://discover.lanl.gov/news/0124-mars-methane-pulses/}.\\
%
\years{2024}``Atmospheric pressure changes could be driving Mars' elusive
methane pulses''.  \emph{LANL News Stories}, 24 January 2024.
\href{https://int.lanl.gov/news/news\_stories/2024/january/0124-mars-methane-pulses.shtml?source=topnews}
{https://int.lanl.gov/news/news\_stories/2024/january/0124-mars-methane-pulses.shtml?source=topnews}
(internal LANL webpage).\\
%
\years{2024}``Atmospheric pressure changes could explain Mars
methane''.\emph{Universe Today}, 29 January 2024.
\href{https://www.universetoday.com/165470/atmosphere-pressure-changes-could-explain-mars-methane/}{https://www.universetoday.com/165470/atmosphere-pressure-changes-could-explain-mars-methane/}.\\
%
%
\years{2024}``NASA's Curiosity Rover Closer to Solving Mystery of Methane Biosignature on Mars, Aiding Search for Life''.
\emph{The Debrief}, 5 February 2024.
\href{https://thedebrief.org/nasas-curiosity-rover-closer-to-solving-mystery-of-methane-biosignature-on-mars-aiding-search-for-life/}{https://thedebrief.org/nasas-curiosity-rover-closer-to-solving-mystery-of-methane-biosignature-on-mars-aiding-search-for-life/}.\\
%
\years{2022}``Study illuminates Mars methane transmission from subsurface
depths that could indicate microbial source''.
\emph{STE Highlights}, August 2022. 
\href{https://www.lanl.gov/science-innovation/science-highlights/2022/2022-08.php#EarthandEnvironmentalSciences-1}
{https://www.lanl.gov/science-innovation/science-highlights/2022/2022-08.php\#EarthandEnvironmentalSciences-1}\\
%
\years{2022}``Study illuminates Mars methane transmission from subsurface''.
\emph{LANL News Stories}, 21 August
2022. \href{https://int.lanl.gov/news/news\_stories/2022/august/0822-mars-methane.shtml}
{https://int.lanl.gov/news/news\_stories/2022/august/0822-mars-methane.shtml}
(internal LANL webpage).\\
%
\years{2022}``Progress on
understanding radioactive gas migration from underground nuclear explosions''. 
\emph{STE Highlights}, February 2022.
\href{https://www.lanl.gov/science-innovation/science-highlights/2022/2022-02.php#EarthandEnvironmentalSciences-1}{https://www.lanl.gov/science-innovation/science-highlights/2022/2022-02.php\#EarthandEnvironmentalSciences-1}\\



\subsection*{Departmental Seminars and Other Talks}

\years{2023} ``Predicting martian methane variations to identify strategic sampling times for the \textit{Curiosity} rover'', LANL, EES-16 SFT Lightning Talk. 1 November,
2023.\\
\years{2022} ``The Mars Underground: Characterizing subsurface methane seepage
on the Red Planet'', Johns Hopkins University, Baltimore, MD.
Environmental Health \& Engineering Seminar. 20 September 2022.\\
\years{2021} ``From Manhattan to Mars: Applying models of subsurface
radionuclide gas seepage from nuclear testing to understand methane release
from the martian subsurface'', Johns Hopkins University, Baltimore, MD.
Environmental Health \& Engineering Seminar. 2 November 2021.\\
\years{2021} ``Determining hydrogeologic properties using well data, barometric
pressures, and tidal analysis'', LANL, EES-16 Science Café series. 12 August 
2021.\\
\years{2021} ``From Manhattan to Mars: Generating novel insights into methane
fluctuations on the Red Planet'', Johns Hopkins University, Baltimore, MD.
Environmental Health \& Engineering Seminar. 2 March 2021.\\
\years{2019} ``Improving estimates of gas transport in fractured rock --
Implications for verification of underground nuclear events'', Johns Hopkins
University, Baltimore, MD. Environmental Health \& Engineering Seminar. 29
October 2019.\\
\years{2020} ``NROMM Seepage Tool: recent capability development and analytical
verification'', LANL, EES-16 Science Café series. 23 July 2020.\\
\years{2019} ``Noble gas diffusion through variably saturated rock --
implications for verification of subsurface nuclear events'', LANL,
EES-16 Science Café series. 8 August 2019.\\
\years{2018} ``The role of fault-zone architectural elements and basement
altered zones on pore pressure propagation and induced seismicity'', LANL,
EES-16 Science Café series. 2 August 2018.\\
\years{2017} ``Improving estimates of subsurface gas transport in unsaturated
fractured media using field tracer data and numerical methods'', LANL, EES-16
Science Café series. 30 November 2017.\\



%--------------------------SERVICE TO PROFESSION-------------------------
% or "Administrative and Collective Duties"
% - journals reviewed for, funding bodies you act as an expert for, evaluation committees on which you sit

\section*{Service Activities and Outreach}

\hypersetup{linkcolor=black,citecolor=blue,filecolor=black,urlcolor=black}

\subsection*{Session Chair}\noindent

\years{2024} \textit{Session Convener, AGU Fall Meeting 2024}\\
``The New Mars Underground: Fluids, Volatiles, and the Future of Mars
Exploration'', Washington DC. 




\subsection*{Journal Reviewer}
%\vspace{-0.9cm}

% SINGLE COLUMN------------  
% \textit{Geophysical Research Letters}, \ \ \textit{Icarus},  \ \ 
% \textit{Journal of Geophysical Research: Solid Earth}, \ \ \textit{Hydrogeology Journal}
    \textit{Geophysical Research Letters}\\
    \textit{Icarus}\\
    \textit{Journal of Geophysical Research: Solid Earth}\\
    \textit{Hydrogeology Journal}

% 3-COLUMN FORMAT ----------  
% \setlength\multicolsep{0pt} %reduce vertical white space at beginning of cols
% % ROW1 --------------------  
% \begin{multicols}{3}
	% % \textit{J. of Geophysical Research}\\
	% \textit{JGR: Planets}\\
	% \textit{Geophysical Research Letters}\\
	% \textit{Hydrogeology Journal}
% \end{multicols}
% % ROW2 --------------------  
% \begin{multicols}{3}
	% \textit{Icarus}\\
	% % \textit{BLANK1}\\
	% % \textit{BLANK2}
% \end{multicols}


\subsection*{Public Outreach}
\setlength{\parskip}{0pt}
\noindent
\years{April 2024} \textit{Los Alamos Middle School Guest Lecturer}\\
Delivered a series of five presentations over two days on measuring atmospheric
methane on Mars with the \textit{Curiosity} rover to several \nth{7} and
\nth{8} grade classes of Life Sciences and Astronomy students at Los Alamos
Middle School. \\ 

\years{April 2023} \textit{Los Alamos Geological Society monthly meeting}\\
Delivered an invited talk (``The Mars Underground: Characterizing subsurface
methane seepage on the Red Planet''; see \href{sec:invitedTalks}{Invited
Talks}) about predicting methane abundance variations at Gale crater, Mars to
members of the Los Alamos Geological Society for their monthly meeting. \\


\subsection*{Advisory Roles}\noindent
\hypersetup{linkcolor=black,citecolor=blue,filecolor=black,urlcolor=blue}

\years{2018 - 2019} \textit{LANL Student Programs Advisory Committee (SPAC)}\\
Served as a 1-year appointment as Graduate Student Representative to a
committee that advises Los Alamos National Laboratory's Student Programs Office
on matters related to the hiring and general well-being of student employees.\\
\href{https://int.lanl.gov/employees/education/spac.shtml}{https://int.lanl.gov/employees/education/spac.shtml}




%--------------------------PROFESSIONAL AFFILIATIONS--------------------------
\section*{Professional Affiliations}
\years{2013 - pres.} \textsc{American Geophysical Union (AGU)}\\
\years{2015 - pres.} \textsc{American Association of Petroleum Geologists (AAPG)}\\
\years{2016 - pres.} \textsc{Geological Society of America (GSA)}\\
\years{2017 - pres.} \textsc{National Ground Water Association (NGWA)}\\
\years{2017 - pres.} \textsc{New Mexico Geological Society (NMGS)}\\
\years{2018 - pres.} \textsc{International Society for Porous Media (InterPore)}



%\vspace{1cm}\usepackage[super]{nth}
\vfill{}
%\hrulefill

\begin{center}
{\scriptsize  Last updated: \today\- •\- Typeset in \href{http://nitens.org/taraborelli/cvtex}{
\fontspec{Times New Roman}\XeTeX }}
\end{center}
% ---- FILL IN THE FULL URL TO YOUR CV HERE
%\href{http://nitens.org/taraborelli/cvtex}{http://nitens.org/taraborelli/cvtex}}
%\end{center}

\end{document}
